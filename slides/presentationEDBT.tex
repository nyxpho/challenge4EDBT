\documentclass[xcolor=dvipsnames]{beamer} 
\usepackage{verbatim}
\usepackage[utf8]{inputenc}

\usetheme{CambridgeUS} 
\usecolortheme{dolphin}
\setbeamertemplate{items}[ball] 
\setbeamertemplate{blocks}[rounded][shadow=true] 
\newcommand{\spara}[1]{\smallskip\noindent{\bf #1}}
\newcommand{\mpara}[1]{\medskip\noindent{\bf #1}}
\newcommand{\para}[1]{\noindent{\bf #1}}
\newcommand{\NPhard}{{\ensuremath{\mathbf{NP}}-hard}}
\newcommand{\bigO}{{\ensuremath{\cal O}}}

\newcommand{\ignore}[1]{}
%\newcommand{\ddense}{$k$-Densest\xspace}


\AtBeginSection[]
{
  \begin{frame}
    \frametitle{Table of Contents}
    \tableofcontents[currentsection]
  \end{frame}
}



%\AtBeginSection[]
%{
%  \begin{frame}
%    \frametitle{Summary}
%    \tableofcontents[currentsection]
%  \end{frame}
%}

\begin{document}


\begin{frame}
\title[Challenge 4]{Similarities in Wikipedia \\ EDBT Summer School 2015}
\author[]{}
\date{4.09.2015}
\maketitle
\end{frame}


\begin{frame}
\frametitle{Table of Contents}
\tableofcontents
\end{frame}


\section{Challenge}

\begin{frame}{Research challenge}
 
\begin{center}
\includegraphics[width=0.25\textwidth, height=0.4\paperheight]{media/wikilogo.eps}
\end{center}

Wikipedia is widely used as a source of information. 

Users start from an initial page (topic). Often, they want to read more related information.

\end{frame}

\begin{frame}{Research challenge}

Pages on Wikipedia link to each other.


However:

\begin{itemize}
\item links might be missing between related articles
\item different pairs of linked articles might be less/more similar
\end{itemize}

\end{frame}


\begin{frame}{Research challenge}

Put an example !!!

\end{frame}

\begin{frame}{Research challenge}


Challenge:

\begin{itemize}
\item compute similarity between articles
\item infer new links between article
\end{itemize}

\end{frame}
\section{Our approach}

\begin{frame}{Content and context}

Similarity between the content of two pages:




\end{frame}

\begin{frame}{Content and context}

Similarity between the context of two pages:

\begin{itemize}
\item related pages link to common pages 
\item related pages are linked by common pages
\item Wikipedia uses categories for grouping together related pages
\end{itemize}

\end{frame}

\begin{frame}{Similarity measure: }



\end{frame}

\begin{frame}{Complexity }



\end{frame}

\section{Results}
\begin{frame}{Implementation}

igraph  PostgreSQL
\end{frame}

\begin{frame}{Results}

\end{frame}


\end{document}
