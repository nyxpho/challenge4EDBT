\documentclass[xcolor=dvipsnames]{beamer} 
\usepackage{verbatim}
\usepackage[utf8]{inputenc}

\usetheme{CambridgeUS} 
\usecolortheme{dolphin}
\setbeamertemplate{items}[ball] 
\setbeamertemplate{blocks}[rounded][shadow=true] 
\newcommand{\spara}[1]{\smallskip\noindent{\bf #1}}
\newcommand{\mpara}[1]{\medskip\noindent{\bf #1}}
\newcommand{\para}[1]{\noindent{\bf #1}}

\newcommand{\noargsim}{sim}
\newcommand{\wsim}[2]{\noargsim(#1,#2)}
\newcommand{\contextSim}[2]{\mathrm{ContextSim(#1,#2)}}
\newcommand{\textSim}[2]{\mathrm{TextSim(#1,#2)}}
\newcommand{\inLinks}[2]{\mathrm{inLinks(#1,#2)}}
\newcommand{\outLinks}[2]{\mathrm{outLinks(#1,#2)}}
\newcommand{\category}[2]{\mathrm{Category(#1,#2)}}
\newcommand{\textfunc}[2]{Func(#1,#2)}
\newcommand{\wikiJack}[3]{Wk_#1(#2,#3)}
\newcommand{\Smthin}[3]{Smth_{#1}(#2,#3)}
\newcommand{\cst}[1]{const_#1}

\newcommand{\ignore}[1]{}

\begin{document}


\begin{frame}
\title[Challenge 4]{Similarities in Wikipedia \\ EDBT Summer School 2015}
\author[]{}
\date{4.09.2015}
\maketitle
\end{frame}





\section{Challenge}

\begin{frame}{Research challenge}
 
\begin{center}
\includegraphics[width=0.25\textwidth, height=0.4\paperheight]{media/wikilogo.eps}
\end{center}

Wikipedia is widely used as a source of information. 

Users start from an initial page (topic). Often, they want to read more related information.

\end{frame}

\begin{frame}{Research challenge}

Pages on Wikipedia link to each other.


However:

\begin{itemize}
\item links might be missing between related articles
\item different pairs of linked articles might be less/more similar
\end{itemize}

\end{frame}


\begin{frame}{Research challenge}

Put an example !!!

\end{frame}

\begin{frame}{Research challenge}


Challenge:

\begin{itemize}
\item compute similarity between articles
\item infer new links between article
\end{itemize}

\end{frame}
\section{Our approach}

\begin{frame}{Content and context}

Similarity between the content of two pages:




\end{frame}

\begin{frame}{Content and context}

Similarity between the context of two pages:

\begin{itemize}
\item related pages link to common pages 
\item related pages are linked by common pages
\item Wikipedia uses categories for grouping together related pages
\end{itemize}

\end{frame}

\begin{frame}{Similarity measure }

We introduce the similarity, $\noargsim$ between two Wikipedia articles $a_1$ and $a_2$ as:

\begin{align*}
	\wsim{a_1}{a_2} =&\ \cst{0} \times \contextSim{a_1}{a_2} \\
	& (1 - \cst{0}) \times \textSim{a_1}{a_2} \\
	\end{align*}


\end{frame}

\begin{frame}{Similarity measure }
	\begin{align*}
	\contextSim{a_1}{a_2} =&\  \cst{1} \times \inLinks{a_1}{a_2}\\
	 	& + \cst{2} \times \outLinks{a_1}{a_2}\\
	 	& + \cst{3} \times \category{a_1}{a_2}\\
	\\
	\Smthin{f}{s1}{s2} =&\ \frac{min(\wikiJack{f}{s1}{s2}, \wikiJack{f}{s2}{s1})}{min(|s1|,|s2|)}\\
\\
	\inLinks{a_1}{a_2} =&\ \Smthin{isSamePage}{inLinks(a)}{inLinks(b)}\\
	\\
	\category{a_1}{a_2} =&\  \text{number of common categories (counting ancestors)}\\
	\\
	\textSim{a_1}{a_2} =&\  Smthing(synonym,Stemmed(a_1),Stemmed(a_2))\\
	\\
	\wikiJack{f}{s1}{s2} =&\ \sum\limits_{a \in s1} \max\limits_{b \in s2}(f(a,b))\\
\end{align*}
\end{frame}

\begin{frame}{Complexity }



\end{frame}

\section{Results}
\begin{frame}{Implementation}

igraph  PostgreSQL
\end{frame}

\begin{frame}{Results}

\end{frame}


\end{document}
